\documentclass[12pt, titlepage]{article}


% Author's up-front packages
\usepackage[T1]{fontenc}
\usepackage[utf8]{inputenc}
\usepackage{longtable}

%Packages from template
\usepackage{amsmath, mathtools}
%\usepackage[round]{natbib}
\usepackage{amsfonts}
\usepackage{amssymb}
\usepackage{graphicx}
\usepackage{colortbl}
\usepackage{xr-hyper}
\usepackage{hyperref}
\usepackage{xfrac}
\usepackage{tabularx}
\usepackage{float}
\usepackage{siunitx}
\usepackage{booktabs}
\usepackage{multirow}
\usepackage[section]{placeins}
\usepackage{caption}
\usepackage{fullpage}

% Author's packages

\usepackage{cite}
\usepackage{indentfirst}
\usepackage{csquotes}
\usepackage{cleveref}

\hypersetup{
bookmarks=true,     % show bookmarks bar?
colorlinks=true,       % false: boxed links; true: colored links
linkcolor=red,          % color of internal links (change box color with linkbordercolor)
citecolor=blue,      % color of links to bibliography
filecolor=magenta,  % color of file links
urlcolor=cyan          % color of external links
}

\usepackage{array}

%% Comments

\usepackage{color}

\newif\ifcomments\commentstrue

\ifcomments
\newcommand{\authornote}[3]{\textcolor{#1}{[#3 ---#2]}}
\newcommand{\todo}[1]{\textcolor{red}{[TODO: #1]}}
\else
\newcommand{\authornote}[3]{}
\newcommand{\todo}[1]{}
\fi

\newcommand{\wss}[1]{\authornote{blue}{SS}{#1}}
\newcommand{\an}[1]{\authornote{magenta}{Author}{#1}}


\newcommand{\progname}{STEM Moir{\'e} GPA}
\externaldocument[SRS:]{../../SRS/SRS}
\externaldocument[TP:]{../../TestPlan/TestPlan}
\externaldocument[MG:]{../MG/MG}

%Set the custom referencing syste
	% Module
\newtheorem{M}{M}
\crefname{M}{M}{Ms}
	% Module Interface Specification
\newtheorem{MIS}{MIS}
\crefname{MIS}{MIS}{MISs}

\begin{document}

\title{Module Interface Specification for \progname{}}

\author{Alexandre Pofelski \\
		macid: pofelska \\
		github: slimpotatoes}

\date{\today}

\maketitle

\pagenumbering{roman}

\section{Revision History}

\begin{tabularx}{\textwidth}{p{3cm}p{2cm}X}
\toprule {\bf Date} & {\bf Version} & {\bf Notes}\\
\midrule
14/11/2017 & 1.0 & First draft \\
\bottomrule
\end{tabularx}

~\newpage

\section{Symbols, Abbreviations and Acronyms}

The same Symbols, Abbreviations and Acronyms as in the SRS, the TestPlan and the MG (available in \href{https://github.com/slimpotatoes/STEM_Moire_GPA}{\progname{}} repository) are used in the Module Interface Specifications document. 

addition to document

\wss{Also add any additional symbols, abbreviations or acronyms}

\newpage

\tableofcontents

\newpage

\pagenumbering{arabic}

\section{Introduction}

The following document details the Module Interface Specifications for \progname{}. The full documentation and implementation can be found in \href{https://github.com/slimpotatoes/STEM_Moire_GPA}{\progname{}} repository.

\section{Notation}

\wss{You should describe your notation.  You can use what is below as
  a starting point.}

The structure of the MIS for modules comes from \cite{HoffmanAndStrooper1995},
with the addition that template modules have been adapted from
\cite{GhezziEtAl2003}.  The mathematical notation comes from Chapter 3 of
\cite{HoffmanAndStrooper1995}.  For instance, the symbol := is used for a
multiple assignment statement and conditional rules follow the form $(c_1
\Rightarrow r_1 | c_2 \Rightarrow r_2 | ... | c_n \Rightarrow r_n )$.

The following table summarizes the primitive data types used by \progname. 

\begin{center}
\renewcommand{\arraystretch}{1.2}
\noindent 
\begin{tabular}{l l p{7.5cm}} 
\toprule 
\textbf{Data Type} & \textbf{Notation} & \textbf{Description}\\ 
\midrule
character & char & a single symbol or digit\\
integer & $\mathbb{Z}$ & an integer number \\
natural number & $\mathbb{N}$ & a natural number \\
real & $\mathbb{R}$ & a real number \\
\bottomrule
\end{tabular} 
\end{center}

\noindent
The specification of \progname{} uses some derived data types: sequences, strings, and
tuples. Sequences are lists filled with elements of the same data type. Strings
are sequences of characters. Tuples contain a list of values, potentially of
different types. In addition, \progname \ uses functions, which
are defined by the data types of their inputs and outputs. Local functions are
described by giving their type signature followed by their specification.

\section{Module Decomposition}

The following table is taken directly from the Module Guide document for this project.

\begin{table}[H]
\centering
\begin{tabular}{p{0.3\textwidth} p{0.6\textwidth}}
\toprule
\textbf{Level 1} & \textbf{Level 2}\\
\midrule

{Hardware-Hiding Module} & ~ \\
\midrule

\multirow{10}{0.3\textwidth}{Behaviour-Hiding Module} & Input\\
& \progname{} Control \\
& \progname{} GUI \\
& User Input \\
& SMH simulation \\
& GPA \\
& Mask \\
& Unstrained region \\
& Conversion \\
& 2D strain tensor \\
\midrule

\multirow{7}{0.3\textwidth}{Software Decision Module} & Fourier Transform \\
& Least square fitting method \\
& Phase calculation \\
& Gradient \\
& Generic GUI/Plot \\
& Data structure \\
& Object structure \\
\bottomrule

\end{tabular}
\caption{Module Hierarchy}
\label{TblMH}
\end{table}

LIST ALL MIS to refer them in other document

\newpage

\section{MIS of Hardware Hiding Module (\texorpdfstring{\cref{MG:M_Hardware}}))} \label{MIS_Hardware}

\subsection{Module}

\wss{Short name for the module}

\subsection{Uses}
Data Structure

\subsection{Syntax}

\subsubsection{Exported Access Programs}

\begin{center}
\begin{tabular}{p{2cm} p{4cm} p{4cm} p{2cm}}
\hline
\textbf{Name} & \textbf{In} & \textbf{Out} & \textbf{Exceptions} \\
\hline
\wss{accessProg} & - & - & - \\
\hline
\end{tabular}
\end{center}

\subsection{Semantics}

\subsubsection{State Variables}


\subsubsection{Access Routine Semantics}

\noindent \wss{accessProg}():
\begin{itemize}
\item transition: \wss{if appropriate} 
\item output: \wss{if appropriate} 
\item exception: \wss{if appropriate} 
\end{itemize}

\section{MIS of STEM Moir{\'e} GPA Control Module (\texorpdfstring{\cref{MG:M_Control}}))} \label{MIS_Control}

\subsection{Module}
main
\subsection{Uses}
\begin{itemize}
\item STEM Moi{\'e} GPA GUI
\item Processing modules \begin{itemize}
	\item Unstrained region 
	\item Conversion
	\item SMH Simulation
	\item GPA
	\item 2D Strain Tensors
\end{itemize}
\item Data Structure
\end{itemize}

\subsection{Syntax}

\subsubsection{Exported Access Programs}

\begin{center}
\begin{tabular}{p{2cm} p{4cm} p{4cm} p{2cm}}
\hline
\textbf{Name} & \textbf{In} & \textbf{Out} & \textbf{Exceptions} \\
\hline
\wss{accessProg} & - & - & - \\
\hline
\end{tabular}
\end{center}

\subsection{Semantics}

\progname{} is designed to have the different steps of the process flow driven by user directly through GUI{\_}SMG. The STEM Moir{\'e} GPA Control Module uses the events in STEM Moi{\'e} GPA GUI to use the processing modules in the order defined by the user.

\subsubsection{State Variables}


\subsubsection{Access Routine Semantics}

\noindent \wss{accessProg}():
\begin{itemize}
\item transition: \wss{if appropriate} 
\item output: \wss{if appropriate} 
\item exception: \wss{if appropriate} 
\end{itemize}

\section{MIS of STEM Moir{\'e} GPA GUI Module (\texorpdfstring{\cref{MG:M_GUISMG}}))} \label{MIS_GUISMG}

\subsection{Module}
GUI{\_}SMG
\subsection{Uses}
\begin{itemize}
\item Generic GUI/Plot
\item Data Structure
\end{itemize}

\subsection{Syntax}

\subsubsection{Exported Access Programs}

\begin{center}
\begin{tabular}{p{2cm} p{4cm} p{4cm} p{2cm}}
\hline
\textbf{Name} & \textbf{In} & \textbf{Out} & \textbf{Exceptions} \\
\hline
\wss{accessProg} & - & - & - \\
\hline
\end{tabular}
\end{center}

\subsection{Semantics}

\progname{} process flow is driven by user through GUI{\_}SMG. User triggers the events that start the wished processing step. 

\subsubsection{State Variables}



\subsubsection{Access Routine Semantics}

\noindent \wss{accessProg}():
\begin{itemize}
\item transition: \wss{if appropriate} 
\item output: \wss{if appropriate} 
\item exception: \wss{if appropriate} 
\end{itemize}

\section{MIS of Imput Module (\texorpdfstring{\cref{MG:M_InputFormat}}))} \label{MIS_Input}

\subsection{Module}
Input
\subsection{Uses}
\begin{itemize}
\item STEM Moi{\'e} GPA GUI
\item Data Structure
\end{itemize}

\subsection{Syntax}

\subsubsection{Exported Access Programs}

\begin{center}
\begin{tabular}{p{2cm} p{4cm} p{4cm} p{2cm}}
\hline
\textbf{Name} & \textbf{In} & \textbf{Out} & \textbf{Exceptions} \\
\hline
\wss{accessProg} & - & - & - \\
\hline
\end{tabular}
\end{center}

\subsection{Semantics}

\subsubsection{State Variables}


\subsubsection{Access Routine Semantics}

\noindent \wss{accessProg}():
\begin{itemize}
\item transition: \wss{if appropriate} 
\item output: \wss{if appropriate} 
\item exception: \wss{if appropriate} 
\end{itemize}

\section{MIS of SMH Simulation (\texorpdfstring{\cref{MG:M_SMHSim}}))} \label{MIS_SHMSim}

\subsection{Module}
SMHSim
\subsection{Uses}
\begin{itemize}
\item Fourier Transform
\item Input
\item Data Structure
\end{itemize}

\subsection{Syntax}

\subsubsection{Exported Access Programs}

\begin{center}
\begin{tabular}{p{2cm} p{4cm} p{4cm} p{2cm}}
\hline
\textbf{Name} & \textbf{In} & \textbf{Out} & \textbf{Exceptions} \\
\hline
\wss{accessProg} & - & - & - \\
\hline
\end{tabular}
\end{center}

\subsection{Semantics}

\subsubsection{State Variables}


\subsubsection{Access Routine Semantics}

\noindent \wss{accessProg}():
\begin{itemize}
\item transition: \wss{if appropriate} 
\item output: \wss{if appropriate} 
\item exception: \wss{if appropriate} 
\end{itemize}

\section{MIS of GPA Module (\texorpdfstring{\cref{MG:M_GPA}}))} \label{MIS_GPA}

\subsection{Module}
GPA
\subsection{Uses}
\begin{itemize}
\item Mask
\item Fourier Transform
\item Phase
\item Gradient
\item Data Structure
\end{itemize}

\subsection{Syntax}

\subsubsection{Exported Access Programs}

\begin{center}
\begin{tabular}{p{2cm} p{4cm} p{4cm} p{2cm}}
\hline
\textbf{Name} & \textbf{In} & \textbf{Out} & \textbf{Exceptions} \\
\hline
\wss{accessProg} & - & - & - \\
\hline
\end{tabular}
\end{center}

\subsection{Semantics}

\subsubsection{State Variables}


\subsubsection{Access Routine Semantics}

\noindent \wss{accessProg}():
\begin{itemize}
\item transition: \wss{if appropriate} 
\item output: \wss{if appropriate} 
\item exception: \wss{if appropriate} 
\end{itemize}

\section{MIS of Mask Module (\texorpdfstring{\cref{MG:M_Mask}}))} \label{MIS_Mask}

\subsection{Module}
GPA
\subsection{Uses}
\begin{itemize}
\item Mask
\item Fourier Transform
\item Phase
\item Gradient
\item Data Structure
\end{itemize}

\subsection{Syntax}

\subsubsection{Exported Access Programs}

\begin{center}
\begin{tabular}{p{2cm} p{4cm} p{4cm} p{2cm}}
\hline
\textbf{Name} & \textbf{In} & \textbf{Out} & \textbf{Exceptions} \\
\hline
\wss{accessProg} & - & - & - \\
\hline
\end{tabular}
\end{center}

\subsection{Semantics}

\subsubsection{State Variables}


\subsubsection{Access Routine Semantics}

\noindent \wss{accessProg}():
\begin{itemize}
\item transition: \wss{if appropriate} 
\item output: \wss{if appropriate} 
\item exception: \wss{if appropriate} 
\end{itemize}

\section{MIS of Unstrained region (\texorpdfstring{\cref{MG:M_URef}}))} \label{MIS_URef}

\subsection{Module}
URef
\subsection{Uses}
\begin{itemize}
\item Least Square Fit
\item Input
\item Data Structure
\end{itemize}

\subsection{Syntax}

\subsubsection{Exported Access Programs}

\begin{center}
\begin{tabular}{p{2cm} p{4cm} p{4cm} p{2cm}}
\hline
\textbf{Name} & \textbf{In} & \textbf{Out} & \textbf{Exceptions} \\
\hline
\wss{accessProg} & - & - & - \\
\hline
\end{tabular}
\end{center}

\subsection{Semantics}

\subsubsection{State Variables}


\subsubsection{Access Routine Semantics}

\noindent \wss{accessProg}():
\begin{itemize}
\item transition: \wss{if appropriate} 
\item output: \wss{if appropriate} 
\item exception: \wss{if appropriate} 
\end{itemize}

\section{MIS of Conversion Module (\texorpdfstring{\cref{MG:M_MtoCConv}}))} \label{MIS_MtoCConv}

\subsection{Module}
MtoCConv
\subsection{Uses}
\begin{itemize}
\item Input
\item Data Structure
\end{itemize}

\subsection{Syntax}

\subsubsection{Exported Access Programs}

\begin{center}
\begin{tabular}{p{2cm} p{4cm} p{4cm} p{2cm}}
\hline
\textbf{Name} & \textbf{In} & \textbf{Out} & \textbf{Exceptions} \\
\hline
\wss{accessProg} & - & - & - \\
\hline
\end{tabular}
\end{center}

\subsection{Semantics}

\subsubsection{State Variables}


\subsubsection{Access Routine Semantics}

\noindent \wss{accessProg}():
\begin{itemize}
\item transition: \wss{if appropriate} 
\item output: \wss{if appropriate} 
\item exception: \wss{if appropriate} 
\end{itemize}

\section{MIS of 2D Strain Tensor Module (\texorpdfstring{\cref{MG:M_StrainCalc}}))} \label{MIS_StrainCalc}

\subsection{Module}
2D{\_}Strain
\subsection{Uses}
Data Structure
\subsection{Syntax}

\subsubsection{Exported Access Programs}

\begin{center}
\begin{tabular}{p{2cm} p{4cm} p{4cm} p{2cm}}
\hline
\textbf{Name} & \textbf{In} & \textbf{Out} & \textbf{Exceptions} \\
\hline
\wss{accessProg} & - & - & - \\
\hline
\end{tabular}
\end{center}

\subsection{Semantics}

\subsubsection{State Variables}


\subsubsection{Access Routine Semantics}

\noindent \wss{accessProg}():
\begin{itemize}
\item transition: \wss{if appropriate} 
\item output: \wss{if appropriate} 
\item exception: \wss{if appropriate} 
\end{itemize}

\section{MIS of Fourier Transform Module (\texorpdfstring{\cref{MG:M_FT}}))} \label{MIS_FT}
\textit{{\#} 2D Fourier transform}
\subsection{Module}
FTCalc
\subsection{Uses}
Data Structure
\subsection{Syntax}

\subsubsection{Exported Access Programs}

\begin{center}
\begin{tabular}{p{2cm} p{4cm} p{4cm} p{2cm}}
\hline
\textbf{Name} & \textbf{In} & \textbf{Out} & \textbf{Exceptions} \\
\hline
$\mathcal{FT}$ & $f:\mathbb{R}^2\rightarrow\mathbb{R}$ & $f:\mathbb{R}^2\rightarrow\mathbb{C}$ & - \\
i$\mathcal{FT}$ & $f:\mathbb{R}^2\rightarrow\mathbb{C}$ & $f:\mathbb{R}^2\rightarrow\mathbb{R}$ & - \\
\hline
\end{tabular}
\end{center}

\subsection{Semantics}

\subsubsection{State Variables}
None

\subsubsection{Access Routine Semantics}

\noindent\textit{{\#} Calculating the 2D Fourier transform of a function $f$} \medskip

\noindent $\mathcal{FT}$($f(x,y)$):
\begin{itemize} 
\item output: $\widetilde{f}(\nu,\mu)$ such that 
\begin{equation*}
\forall (\nu,\mu) \in \mathbb{R}^2 \wedge \forall (x,y) \in \mathbb{R}^2, \ \widetilde{f}(\nu,\mu)\int\int_{-\infty}^{\infty}f(x,y)e^{-2i\pi(\nu x+\mu y)}dxdy
\end{equation*}
\item exception: \wss{if appropriate} 
\end{itemize}

\noindent\textit{{\#} Calculating the inverse 2D Fourier transform of a function $\widetilde{f}$}\medskip

\noindent i$\mathcal{FT}$($\widetilde{f}(\nu,\mu)$):
\begin{itemize} 
\item output: $f(x,y)$ such that 
\begin{equation*}
\forall (x,y) \in \mathbb{R}^2 \wedge \forall (\nu,\mu) \in \mathbb{R}^2 , \ f(x,y)=\int\int_{-\infty}^{\infty}\widetilde{f}(\nu,\mu)e^{2i\pi(\nu x+\mu y)}dxdy
\end{equation*}
\item exception: \wss{if appropriate} 
\end{itemize}

\section{MIS of Gradient Module (\texorpdfstring{\cref{MG:M_Gradient}}))} \label{MIS_Gradient}

\subsection{Module}
Gradient
\subsection{Uses}
Data Structure
\subsection{Syntax}

\subsubsection{Exported Access Programs}

\begin{center}
\begin{tabular}{p{2cm} p{4cm} p{4cm} p{2cm}}
\hline
\textbf{Name} & \textbf{In} & \textbf{Out} & \textbf{Exceptions} \\
\hline
\wss{accessProg} & - & - & - \\
\hline
\end{tabular}
\end{center}

\subsection{Semantics}

\subsubsection{State Variables}


\subsubsection{Access Routine Semantics}

\noindent \wss{accessProg}():
\begin{itemize}
\item transition: \wss{if appropriate} 
\item output: \wss{if appropriate} 
\item exception: \wss{if appropriate} 
\end{itemize}

\section{MIS of Least Square Fit Method Module (\texorpdfstring{\cref{MG:M_LSFM}}))} \label{MIS_LSFM}

\subsection{Module}
LSFM
\subsection{Uses}
Data Structure
\subsection{Syntax}

\subsubsection{Exported Access Programs}

\begin{center}
\begin{tabular}{p{2cm} p{4cm} p{4cm} p{2cm}}
\hline
\textbf{Name} & \textbf{In} & \textbf{Out} & \textbf{Exceptions} \\
\hline
\wss{accessProg} & - & - & - \\
\hline
\end{tabular}
\end{center}

\subsection{Semantics}

\subsubsection{State Variables}


\subsubsection{Access Routine Semantics}

\noindent \wss{accessProg}():
\begin{itemize}
\item transition: \wss{if appropriate} 
\item output: \wss{if appropriate} 
\item exception: \wss{if appropriate} 
\end{itemize}


\section{MIS of Phase Operation Module (\texorpdfstring{\cref{MG:M_Phase}}))} \label{MIS_Phase}

\subsection{Module}
PhaseCalc
\subsection{Uses}
Data Structure
\subsection{Syntax}

\subsubsection{Exported Access Programs}

\begin{center}
\begin{tabular}{p{2cm} p{4cm} p{4cm} p{2cm}}
\hline
\textbf{Name} & \textbf{In} & \textbf{Out} & \textbf{Exceptions} \\
\hline
\wss{accessProg} & - & - & - \\
\hline
\end{tabular}
\end{center}

\subsection{Semantics}

\subsubsection{State Variables}


\subsubsection{Access Routine Semantics}

\noindent \wss{accessProg}():
\begin{itemize}
\item transition: \wss{if appropriate} 
\item output: \wss{if appropriate} 
\item exception: \wss{if appropriate} 
\end{itemize}


\section{MIS of Data Structure Module (\texorpdfstring{\cref{MG:M_DataStruct}}))} \label{MIS_DataStruct}

\subsection{Module}
DataStruct
\subsection{Uses}

\subsection{Syntax}

\subsubsection{Exported Access Programs}

\begin{center}
\begin{tabular}{p{2cm} p{4cm} p{4cm} p{2cm}}
\hline
\textbf{Name} & \textbf{In} & \textbf{Out} & \textbf{Exceptions} \\
\hline
set & Metadata & - & - \\
\hline
\end{tabular}
\end{center}

\subsection{Semantics}

\subsubsection{State Variables}


\subsubsection{Access Routine Semantics}

\noindent \wss{accessProg}():
\begin{itemize}
\item transition: \wss{if appropriate} 
\item output: \wss{if appropriate} 
\item exception: \wss{if appropriate} 
\end{itemize}

\section{MIS of Generic GUI/Plot Module (\texorpdfstring{\cref{MG:M_GUIGene}}))} \label{MIS_GUIGene}

\subsection{Module}
GUIGene
\subsection{Uses}
Hardware-Hiding
Data Structure
\subsection{Syntax}

\subsubsection{Exported Access Programs}

\begin{center}
\begin{tabular}{p{2cm} p{4cm} p{4cm} p{2cm}}
\hline
\textbf{Name} & \textbf{In} & \textbf{Out} & \textbf{Exceptions} \\
\hline
\wss{accessProg} & - & - & - \\
\hline
\end{tabular}
\end{center}

\subsection{Semantics}

\subsubsection{State Variables}


\subsubsection{Access Routine Semantics}

\noindent \wss{accessProg}():
\begin{itemize}
\item transition: \wss{if appropriate} 
\item output: \wss{if appropriate} 
\item exception: \wss{if appropriate} 
\end{itemize}


\newpage

\bibliographystyle{ieeetr}
\bibliography{MG}

\newpage

\section{Appendix} \label{Appendix}

\wss{Extra information if required}

\end{document}