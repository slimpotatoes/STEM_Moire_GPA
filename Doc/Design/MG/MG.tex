\documentclass[12pt, titlepage]{article}

% Author's up-front packages
\usepackage[T1]{fontenc}
\usepackage[utf8]{inputenc}
\usepackage{longtable}

%Packages on template
\usepackage{fullpage}
\usepackage{multirow}
\usepackage{booktabs}
\usepackage{tabularx}
\usepackage{graphicx}
\usepackage{float}
% Personal addtion
\usepackage{xr-hyper}
% End Personal addtion
\usepackage{hyperref}

% Author's packages

\usepackage{amsmath, mathtools}
\usepackage{amsfonts}
\usepackage{amssymb}
\usepackage{graphicx}
\usepackage{cite}
\usepackage{indentfirst}
\usepackage{float}
\usepackage{csquotes}
\usepackage{cleveref}

%Hypersetup on template
\hypersetup{
    colorlinks,
    citecolor=black,
    filecolor=black,
    linkcolor=red,
    urlcolor=blue
}

\newcommand{\progname}{STEM Moir{\'e} GPA}
\externaldocument[SRS:]{../../SRS/SRS}
\externaldocument[TP:]{../../TestPlan/TestPlan}

% Author choice to remove
% \usepackage[round]{natbib}

\input{../../Comments}

\newcounter{acnum}
\newcommand{\actheacnum}{AC\theacnum}
\newcommand{\acref}[1]{AC\ref{#1}}

\newcounter{ucnum}
\newcommand{\uctheucnum}{UC\theucnum}
\newcommand{\uref}[1]{UC\ref{#1}}

\newcounter{mnum}
\newcommand{\mthemnum}{M\themnum}
\newcommand{\mref}[1]{M\ref{#1}}


%Set the custom referencing system (author's initiative)
	% Anticipated Changes
\newtheorem{AC}{AC}
\crefname{AC}{AC}{ACs}
	% Unlikely Changes
\newtheorem{UC}{UC}
\crefname{UC}{UC}{UCs}
	% Module
\newtheorem{M}{M}
\crefname{M}{M}{Ms}


\begin{document}

\title{Module Guide (MG) \\
STEM Moir{\'e} GPA} 
\author{Alexandre Pofelski \\
		macid: pofelska \\
		github: slimpotatoes}
\date{\today}

\maketitle

\pagenumbering{roman}

\section{Revision History}

\begin{table}[h]
\caption{\bf Revision History}
\begin{tabularx}{\textwidth}{p{3cm}p{2cm}X}
\toprule {\bf Date} & {\bf Version} & {\bf Notes}\\
\midrule
31/10/2017 & 1.0 & First Draft\\
\bottomrule
\end{tabularx}
\end{table}

\newpage

\tableofcontents

\listoftables

\listoffigures

\newpage

\pagenumbering{arabic}

\section{Introduction}

%Keep as comment
\iffalse
Decomposing a system into modules is a commonly accepted approach to developing
software.  A module is a work assignment for a programmer or programming
team~\cite{ParnasEtAl1984}.  We advocate a decomposition
based on the principle of information hiding~\cite{Parnas1972a}.  This
principle supports design for change, because the ``secrets'' that each module
hides represent likely future changes.  Design for change is valuable in SC,
where modifications are frequent, especially during initial development as the
solution space is explored.  

Our design follows the rules layed out by \cite{ParnasEtAl1984}, as follows:
\begin{itemize}
\item System details that are likely to change independently should be the
  secrets of separate modules.
\item Each data structure is used in only one module.
\item Any other program that requires information stored in a module's data
  structures must obtain it by calling access programs belonging to that module.
\end{itemize}

After completing the first stage of the design, the Software Requirements
Specification (SRS), the Module Guide (MG) is developed~\cite{ParnasEtAl1984}. The MGspecifies the modular structure of the system and is intended to allow bothdesigners and maintainers to easily identify the parts of the software. The
potential readers of this document are as follows:

\begin{itemize}
\item New project members: This document can be a guide for a new project member
  to easily understand the overall structure and quickly find the
  relevant modules they are searching for.
\item Maintainers: The hierarchical structure of the module guide improves the
  maintainers' understanding when they need to make changes to the system. It is
  important for a maintainer to update the relevant sections of the document
  after changes have been made.
\item Designers: Once the module guide has been written, it can be used to
  check for consistency, feasibility and flexibility. Designers can verify the
  system in various ways, such as consistency among modules, feasibility of the
  decomposition, and flexibility of the design.
\end{itemize}
\fi
%End comment

The module guide document is providing to the reader the decomposition of \progname{} into smaller pieces to help the implementation phase. The decomposition is based on the principle of information hiding~\cite{Parnas1972a} which interest relies on its flexibility to design change. The module design follows the rules layed out by \cite{ParnasEtAl1984}, as follows:
\begin{itemize}
\item System details that are likely to change independently should be the
  secrets of separate modules.
\item Each data structure is used in only one module.
\item Any other program that requires information stored in a module's data
  structures must obtain it by calling access programs belonging to that module.
\end{itemize}

The module guide (MG) follows the Software Requirements Specification (SRS) available in the documentation tree. Terminologies, symbols and acronyms used in the document are described in the SRS and TestPlan documents. The rest of the document is organized as follows. \Cref{SecChange} lists the anticipated and unlikely changes of the software requirements. \Cref{SecMH} summarizes the module decomposition that was constructed according to the likely changes. \Cref{SecConnection} specifies the connections between the software requirements and the modules. \Cref{SecMD} gives a detailed description of the modules. Section \cref{SecTM} includes two traceability matrices. One checks the completeness of the design against the requirements provided in the SRS. The other shows the relation between anticipated changes and the modules. \Cref{SecUse} describes the use relation between modules.

\section{Anticipated and Unlikely Changes} \label{SecChange}

This section lists possible changes to the system. According to the likeliness
of the change, the possible changes are classified into two
categories. Anticipated changes are listed in \cref{SecAchange}, and
unlikely changes are listed in \cref{SecUchange}.

\subsection{Anticipated Changes} \label{SecAchange}

Anticipated changes are the source of the information that is to be hidden
inside the modules. Ideally, changing one of the anticipated changes will only
require changing the one module that hides the associated decision. 

\begin{AC}\normalfont Hardware running \progname{}
\label{AC_Hardware}
\end{AC}

\begin{AC}\normalfont SMH, ICREF, inputs file format
\label{AC_FormatFile}
\end{AC}

\begin{AC}\normalfont Small strain and small gradient of strain assumption
\label{AC_Assum_SmallStrain}
\end{AC}

\begin{AC}\normalfont SMH simulation from a reference algorithm
\label{AC_Assum_2DPeriodicGrid}
\end{AC}

\begin{AC}\normalfont GPA algorithm
\label{AC_GPA_algo}
\end{AC}

\begin{AC}\normalfont Mask function for frequency isolation (in GPA)
\label{AC_GPA_mask}
\end{AC}

\begin{AC}\normalfont Fitting algorithm to get unstrained reference (in GPA)
\label{AC_GPA_RefFit}
\end{AC}

\begin{AC}\normalfont Output format
\label{AC_Export}
\end{AC}

\begin{AC}\normalfont Data plotting
\label{AC_Plot}
\end{AC}

\iffalse
\begin{description}
\item[\refstepcounter{acnum} \actheacnum \label{acHardware}:] The specific
  hardware on which the software is running.
\item[\refstepcounter{acnum} \actheacnum \label{acInput}:] The format of the
  initial input data.
\item ...
\end{description}
\fi

\subsection{Unlikely Changes} \label{SecUchange}

\iffalse
The module design should be as general as possible. However, a general system is
more complex. Sometimes this complexity is not necessary. Fixing some design
decisions at the system architecture stage can simplify the software design. If
these decision should later need to be changed, then many parts of the design
will potentially need to be modified. Hence, it is not intended that these
decisions will be changed.
\fi

The unlikely changes are design decisions fixing some aspects of \progname{}. The lack of modularity is compensated by a simplification in the software design. If any of the elements are modified, important changes would be expected in the design. Therefore the following decisions are expected to stay unchanged.

\begin{UC}\normalfont SMH data type (2D arrays)
\label{UC_Input_data}
\end{UC}

\begin{UC}\normalfont Perfect 2D periodic scanning grid assumption
\label{UC_Assum_2DPeriodicGrid}
\end{UC}

\begin{UC}\normalfont Pixel data type (real number)
\label{UC_Assum_Pixel}
\end{UC}

\begin{UC}\normalfont Keyboard/mouse interface device with user
\label{UC_Assum_Pixel}
\end{UC}

\iffalse
\begin{description}
\item[\refstepcounter{ucnum} \uctheucnum \label{ucIO}:] Input/Output devices
  (Input: File and/or Keyboard, Output: File, Memory, and/or Screen).
\item[\refstepcounter{ucnum} \uctheucnum \label{ucInput}:] There will always be
  a source of input data external to the software.
\item ...
\end{description}
\fi

\section{Module Hierarchy} \label{SecMH}

This section provides an overview of the module design. Modules are summarized
in a hierarchy decomposed by secrets in \cref{TblMH}. The modules listed
below, which are leaves in the hierarchy tree, are the modules that will
actually be implemented.

\iffalse
\begin{description}
\item [\refstepcounter{mnum} \mthemnum \label{mHH}:] Hardware-Hiding Module
\item ...
\end{description}
\fi

\begin{M}\normalfont Hardware Module
\label{M_Hardware}
\end{M}

\begin{M}\normalfont Input Module
\label{M_InputFormat}
\end{M}

\begin{M}\normalfont Output Module
\label{M_Output}
\end{M}

\begin{M}\normalfont Fourier Transform Module
\label{M_FT}
\end{M}

\begin{M}\normalfont Mask Module
\label{M_Mask}
\end{M}

\begin{M}\normalfont Gradient Module
\label{M_Gradient}
\end{M}

\begin{M}\normalfont Least Square Fitting Method Module
\label{M_LSFM}
\end{M}

\begin{M}\normalfont Phase operation Module
\label{M_Phase}
\end{M}

\begin{M}\normalfont Strain calculation Module
\label{M_StrainCalc}
\end{M}

\begin{M}\normalfont Plotting Module
\label{M_Plot}
\end{M}

\begin{M}\normalfont GUI Module
\label{M_GUI}
\end{M}

\begin{M}\normalfont GPA Module
\label{M_GPA}
\end{M}

\begin{M}\normalfont SMH simulation Module
\label{M_SMHSim}
\end{M}

\begin{M}\normalfont Unstrained reference calculation Module
\label{M_URef}
\end{M}

\begin{M}\normalfont Moir{\'e} to Crystal conversion Module
\label{M_MtoCConv}
\end{M}

\begin{M}\normalfont Control Module
\label{M_Control}
\end{M}

\begin{M}\normalfont Data Structure Module
\label{M_DataStruct}
\end{M}

\begin{table}[H]
\centering
\begin{tabular}{p{0.3\textwidth} p{0.3\textwidth} p{0.3\textwidth}}
\toprule
\textbf{Level 1} & \textbf{Level 2} & \textbf{Level 3}\\
\midrule

{Hardware-Hiding Module} & ~ \\
\midrule

\multirow{12}{0.3\textwidth}{Behaviour-Hiding Module} & Input & \\
& Output &\\
& Control Module &\\
& SMH Simulation & Fourier Transform \\
& Unstrained reference calculation & Least Square Fitting Method \\
& Moir{\'e} to Crystal conversion &\\
& Strain calculation &\\ 
& \multirow{4}{0.3\textwidth}{GPA} & Fourier Transform \newline Mask \newline Phase operation \newline Gradient\\
\midrule

\multirow{3}{0.3\textwidth}{Software Decision Module} & Data StructureSS &\\
& GUI & \\
& Plotting & \\
\bottomrule

\end{tabular}
\caption{Module Hierarchy}
\label{TblMH}
\end{table}

\section{Connection Between Requirements and Design} \label{SecConnection}

The design of the system is intended to satisfy the requirements developed in
the SRS. In this stage, the system is decomposed into modules. The connection
between requirements and modules is listed in Table \ref{TblRT}.

\section{Module Decomposition} \label{SecMD}

Modules are decomposed according to the principle of ``information hiding''
proposed by \cite{ParnasEtAl1984}. The \emph{Secrets} field in a module
decomposition is a brief statement of the design decision hidden by the
module. The \emph{Services} field specifies \emph{what} the module will do
without documenting \emph{how} to do it. For each module, a suggestion for the
implementing software is given under the \emph{Implemented By} title. If the
entry is \emph{OS}, this means that the module is provided by the operating
system or by standard programming language libraries.  Also indicate if the
module will be implemented specifically for the software.

Only the leaf modules in the
hierarchy have to be implemented. If a dash (\emph{--}) is shown, this means
that the module is not a leaf and will not have to be implemented. Whether or
not this module is implemented depends on the programming language
selected.

\subsection{Hardware Hiding Modules (\cref{M_Hardware})}

\begin{description}
\item[Secrets:]The data structure and algorithm used to implement the virtual
  hardware.
\item[Services:]Serves as a virtual hardware used by the rest of the
  system. This module provides the interface between the hardware and the
  software. So, the system can use it to display outputs or to accept inputs.
\item[Implemented By:] OS
\end{description}

\subsection{Behaviour-Hiding Module}

\begin{description}
\item[Secrets:]The contents of the required behaviours.
\item[Services:]Includes programs that provide externally visible behaviour of
  the system as specified in the software requirements specification (SRS)
  documents. This module serves as a communication layer between the
  hardware-hiding module and the software decision module. The programs in this
  module will need to change if there are changes in the SRS.
\item[Implemented By:] --
\end{description}

\subsubsection{Input Module (\cref{M_InputFormat})}

\begin{description}
\item[Secrets:]The format and structure of the input data.
\item[Services:]Converts the input data into the data structure used by the
  input parameters module.
\item[Implemented By:] \progname{}
\end{description}

\subsubsection{Output Module (\cref{M_Output})}

\begin{description}
\item[Secrets:]Output format 
\item[Services:]Transforms data to be usable by the plotter and the hardware device
\item[Implemented By:] \progname{}
\end{description}

\subsubsection{Fourier Transform Module (\cref{M_FT})}

\begin{description}
\item[Secrets:] Fourier transform algorithm
\item[Services:] Transforms data into its Fourier transform
\item[Implemented By:] External Python library
\end{description}

\subsubsection{Least Square Fitting Method Module (\cref{M_LSFM})}

\begin{description}
\item[Secrets:] Least Square Fitting algorithm
\item[Services:] Fit set of data into a 2D linear function using the Least squares method
\item[Implemented By:] External Python library
\end{description}

\subsubsection{Gradient Module (\cref{M_Gradient})}

\begin{description}
\item[Secrets:] 2D Gradient algorithm
\item[Services:] Perform the 2D derivative of a data set.
\item[Implemented By:] External Python library
\end{description}

\subsubsection{Phase Module (\cref{M_Phase})}

\begin{description}
\item[Secrets:] Algorithm to wrap/unwrap the phase
\item[Services:] Wrap/unwrap the phase by adding/removing discontinuity in the interval [0,2$\pi$]
\item[Implemented By:] External Python library
\end{description}

\subsubsection{SMH simulation Module (\cref{M_SMHSim})}

\begin{description}
\item[Secrets:] Algorithm to simulate the SMH from the reference image
\item[Services:] Simulate the SMH in Fourier space using the pixel size $p$ of $I_{\mathit{SMH}_{\texttt{exp}}}$ using the Fourier transform the image reference $I_{C_{\texttt{ref}}}$
\item[Implemented By:] \progname{}
\end{description}

\subsubsection{Strain calculation Module (\cref{M_StrainCalc})}

\begin{description}
\item[Secrets:] Algorithm to calculate the strain from distortion matrix
\item[Services:] Calculate the strain level on each pixel of the data
\item[Implemented By:] \progname{}
\end{description}

\subsubsection{Control Module (\cref{M_Control})}

\begin{description}
\item[Secrets:] Execution flow of \progname{}
\item[Services:] Calls the different modules in the appropriate order
\item[Implemented By:] \progname{}
\end{description}

\subsubsection{Moir{\'e} to Crystal conversion Module (\cref{M_MtoCConv})}

\begin{description}
\item[Secrets:] Algorithm to convert a Moir{\'e} wave vector to a crystalline wave vector
\item[Services:] Affine vectorial transformation using the Moir{\'e} data from GPA and the sampling vectors user inputs.
\item[Implemented By:] \progname{}
\end{description}

\subsubsection{Unstrained reference calculation Module (\cref{M_URef})}

\begin{description}
\item[Secrets:] Algorithm to calculate the unstrained reference
\item[Services:] Calculation of the unstrained reference based on the user input $U$ and the result of the fitting method.
\item[Implemented By:] \progname{}
\end{description}

\subsubsection{Mask Module (\cref{M_Mask})}

\begin{description}
\item[Secrets:] Algorithm to isolate a portion of an image
\item[Services:] Function isolating and weighting a sub area of an image
\item[Implemented By:] \progname{}
\end{description}

\subsubsection{GPA Module (\cref{M_GPA})}

\begin{description}
\item[Secrets:] GPA method
\item[Services:] Map the variation of a vector in Fourier space related the the variation of the periodicity in real space
\item[Implemented By:] \progname{}
\end{description}


\subsection{Software Decision Module}

\subsubsection{Plotting Module (\cref{M_Plot})}

\begin{description}
\item[Secrets:] Method to transforming the data format into plot format
\item[Services:] Transform data into format usable by the hardware to display data into readable format for the user
 
\item[Implemented By:] External Python library
\end{description}

\subsubsection{GUI Module (\cref{M_GUI})}

\begin{description}
\item[Secrets:] Method to interact with the user
\item[Services:] Provide an interactive interface for the user to feed the information needed for the proper execution of \progname{}. 
 
\item[Implemented By:] External Python library
\end{description}

\subsubsection{Data Structure Module (\cref{M_DataStruct})}

\begin{description}
\item[Secrets:] Data format for an image
\item[Services:] Provide convenient format to store, read and manipulate all elements (pixel) from an image
 
\item[Implemented By:] External Python library
\end{description}



\section{Traceability Matrix} \label{SecTM}

This section shows two traceability matrices: between the modules and the
requirements and between the modules and the anticipated changes.

% the table should use mref, the requirements should be named, use something
% like fref
\begin{table}[H]
\centering
\begin{tabular}{p{0.2\textwidth} p{0.6\textwidth}}
\toprule
\textbf{Req.} & \textbf{Modules}\\
\midrule
R1 & \mref{mHH}, \mref{mInput}, \mref{mParams}, \mref{mControl}\\
R2 & \mref{mInput}, \mref{mParams}\\
R3 & \mref{mVerify}\\
R4 & \mref{mOutput}, \mref{mControl}\\
R5 & \mref{mOutput}, \mref{mODEs}, \mref{mControl}, \mref{mSeqDS}, \mref{mSolver}, \mref{mPlot}\\
R6 & \mref{mOutput}, \mref{mODEs}, \mref{mControl}, \mref{mSeqDS}, \mref{mSolver}, \mref{mPlot}\\
R7 & \mref{mOutput}, \mref{mEnergy}, \mref{mControl}, \mref{mSeqDS}, \mref{mPlot}\\
R8 & \mref{mOutput}, \mref{mEnergy}, \mref{mControl}, \mref{mSeqDS}, \mref{mPlot}\\
R9 & \mref{mVerifyOut}\\
R10 & \mref{mOutput}, \mref{mODEs}, \mref{mControl}\\
R11 & \mref{mOutput}, \mref{mODEs}, \mref{mEnergy}, \mref{mControl}\\
\bottomrule
\end{tabular}
\caption{Trace Between Requirements and Modules}
\label{TblRT}
\end{table}

\begin{table}[H]
\centering
\begin{tabular}{p{0.2\textwidth} p{0.6\textwidth}}
\toprule
\textbf{AC} & \textbf{Modules}\\
\midrule
\acref{acHardware} & \mref{mHH}\\
\acref{acInput} & \mref{mInput}\\
\acref{acParams} & \mref{mParams}\\
\acref{acVerify} & \mref{mVerify}\\
\acref{acOutput} & \mref{mOutput}\\
\acref{acVerifyOut} & \mref{mVerifyOut}\\
\acref{acODEs} & \mref{mODEs}\\
\acref{acEnergy} & \mref{mEnergy}\\
\acref{acControl} & \mref{mControl}\\
\acref{acSeqDS} & \mref{mSeqDS}\\
\acref{acSolver} & \mref{mSolver}\\
\acref{acPlot} & \mref{mPlot}\\
\bottomrule
\end{tabular}
\caption{Trace Between Anticipated Changes and Modules}
\label{TblACT}
\end{table}

\section{Use Hierarchy Between Modules} \label{SecUse}

In this section, the uses hierarchy between modules is
provided. \cite{Parnas1978} said of two programs A and B that A {\em uses} B if
correct execution of B may be necessary for A to complete the task described in
its specification. That is, A {\em uses} B if there exist situations in which
the correct functioning of A depends upon the availability of a correct
implementation of B.  Figure \ref{FigUH} illustrates the use relation between
the modules. It can be seen that the graph is a directed acyclic graph
(DAG). Each level of the hierarchy offers a testable and usable subset of the
system, and modules in the higher level of the hierarchy are essentially simpler
because they use modules from the lower levels.

\begin{figure}[H]
\centering
\includegraphics[width=\linewidth]{Figure_Hierarchy.png}
\caption{Use hierarchy among modules}
\label{FigUH}
\end{figure}

%\section*{References}

\bibliographystyle{ieeetr}
\bibliography{MG}

\end{document}