\documentclass[12pt]{article}

% Author's up-front packages
\usepackage[T1]{fontenc}
\usepackage[utf8]{inputenc}

% Packages on template
\usepackage{amsmath, mathtools}
\usepackage{amsfonts}
\usepackage{amssymb}
\usepackage{graphicx}
\usepackage{colortbl}
\usepackage{xr}
\usepackage{hyperref}
\usepackage{longtable}
\usepackage{xfrac}
\usepackage{tabularx}
\usepackage{float}
\usepackage{siunitx}
\usepackage{booktabs}
\usepackage{caption}
\usepackage{pdflscape}
\usepackage{afterpage}

% ---- Author's choice to remove them ----
%\usepackage[round]{natbib}
%\usepackage{refcheck}

% Author's packages
\usepackage{cite}
\usepackage{indentfirst}
\usepackage{cleveref}
\usepackage{float}

\hypersetup{
    %bookmarks=true,         % show bookmarks bar?
    colorlinks=true,       % false: boxed links; true: colored links
    linkcolor=red,          % color of internal links (change box color with linkbordercolor)
    citecolor=green,        % color of links to bibliography
    filecolor=magenta,      % color of file links
    urlcolor=blue           % color of external links
}

\input{../Comments}

% For easy change of table widths
\newcommand{\colZwidth}{1.0\textwidth}
\newcommand{\colAwidth}{0.13\textwidth}
\newcommand{\colBwidth}{0.82\textwidth}
\newcommand{\colCwidth}{0.1\textwidth}
\newcommand{\colDwidth}{0.05\textwidth}
\newcommand{\colEwidth}{0.8\textwidth}
\newcommand{\colFwidth}{0.17\textwidth}
\newcommand{\colGwidth}{0.5\textwidth}
\newcommand{\colHwidth}{0.28\textwidth}


% Used so that cross-references have a meaningful prefix
\newcommand{\progname}{STEM Moir{\'e} GPA}

\usepackage{fullpage}

\begin{document}

\title{STEM Moir{\'e} GPA} 
\author{Alexandre Pofelski \\
		macid: pofelska \\
		github: slimpotatoes}
\date{\today}
	
\maketitle

\pagenumbering{roman}
\tableofcontents

\begin{table}[bp]
\caption{\bf Revision History}
\begin{tabularx}{\textwidth}{p{3cm}p{2cm}X}
\toprule {\bf Date} & {\bf Version} & {\bf Notes}\\
\midrule
19/09/2017 & 1.0 & First Draft\\
\bottomrule
\end{tabularx}
\end{table}

\clearpage

\section{Reference Material}

\subsection{Table of Units}

Throughout this document SI (\href{<https://physics.nist.gov/cuu/Units/index.html>}{Syst\`{e}me Internationale d'Unit\'{e}s}) is employed as the unit system. In addition to the basic units, several derived units are used as described below.  For each unit, the symbol is given followed by a description of the unit and the SI name.\par \bigskip

\renewcommand{\arraystretch}{1.2}
%\begin{table}[ht]
  \noindent \begin{tabular}{l l l} 
    \toprule		
    \textbf{Symbol} & \textbf{Base quantity} & \textbf{Name SI}\\
    \midrule 
    \si{\metre} & length & metre\\
    \si{\per\metre} & reciprocal meter & wave number\\
    \bottomrule
  \end{tabular}
  %	\caption{Provide a caption}
%\end{table}

\wss{Only include the units that your SRS actually uses}

\subsection{Table of Symbols}

The table that follows summarizes the symbols used in this document along with
their units.  The symbols are listed in alphabetical order.

\renewcommand{\arraystretch}{1.2}
%\noindent \begin{tabularx}{1.0\textwidth}{l l X}
\noindent \begin{longtable*}{l l p{12cm}} \toprule
\textbf{Symbol} & \textbf{Unit} & \textbf{Description}\\
\midrule 
$A_C$ & \si[per-mode=symbol] {\square\metre} & coil surface area
\\
$A_\text{in}$ & \si[per-mode=symbol] {\square\metre} & surface area over 
which heat is transferred in
\\ 
\bottomrule
\end{longtable*}
\wss{Use your problems actual symbols.  The si package is a good idea to use for
  units.}

\subsection{Abbreviations and Acronyms}

\renewcommand{\arraystretch}{1.2}
\begin{tabular}{l l} 
  \toprule		
  \textbf{symbol} & \textbf{description}\\
  \midrule 
  A & Assumption\\
  DD & Data Definition\\
  GD & General Definition\\
  GS & Goal Statement\\
  IM & Instance Model\\
  LC & Likely Change\\
  PS & Physical System Description\\
  R & Requirement\\
  SRS & Software Requirements Specification\\
  \progname{} & \wss{put your program name here}\\
  T & Theoretical Model\\
  \bottomrule
\end{tabular}\\

\wss{Add any other abbreviations or acronyms that you add}

\newpage
\pagenumbering{arabic}

\section{Introduction}

\wss{This SRS template is based on \citet{SmithAndLai2005, SmithEtAl2007}.  It
  will get you started, but you will have to make changes.  Any changes to
  section headings should be approved by the instructor, since that implies a
  deviation from the template.  Although the bits shown below do not include
  type information, you may need to add this information for your problem.}

\wss{Feel free to change the appearance of the report by modifying the LaTeX
  commands.}

\wss{If you are documenting a family of models, you can start from this same
  template, but you will have to add a section for variabilities.  For program
  families you should look at \cite{Smith2006, SmithMcCutchanAndCarette2017}.
  You should be able to do one document that captures the commonality analysis
  and the requirements.}

\subsection{Purpose of Document}

\subsection{Scope of Requirements} 

\subsection{Characteristics of Intended Reader} 

\subsection{Organization of Document}

\section{General System Description}

This section identifies the interfaces between the system and its environment,
describes the user characteristics and lists the system constraints.

\subsection{System Context}

\wss{Your system context will likely include an explicit list of user and system
  responsibilities}

\begin{itemize}
\item User Responsibilities:
\begin{itemize}
\item 
\end{itemize}
\item \progname{} Responsibilities:
\begin{itemize}
\item Detect data type mismatch, such as a string of characters instead of a
  floating point number
\item 
\end{itemize}
\end{itemize}

\subsection{User Characteristics} \label{SecUserCharacteristics}

The end user of \progname{} should have an understanding of undergraduate Level
1 Calculus and Physics.

\subsection{System Constraints}

\wss{You may not have any system constraints}

\section{Specific System Description}

This section first presents the problem description, which gives a high-level
view of the problem to be solved.  This is followed by the solution characteristics
specification, which presents the assumptions, theories, definitions and finally
the instance models.  \wss{Add any project specific details that are relevant
  for the section overview.}

\subsection{Problem Description} \label{Sec_pd}

\progname{} is \wss{what problem does your program solve?}

\subsubsection{Terminology and  Definitions}

This subsection provides a list of terms that are used in the subsequent
sections and their meaning, with the purpose of reducing ambiguity and making it
easier to correctly understand the requirements:

\begin{itemize}

\item 

\end{itemize}

\subsubsection{Physical System Description}

The physical system of \progname{}, as shown in Figure~?,
includes the following elements:


\wss{A figure here may make sense for most SRS documents}

% \begin{figure}[h!]
% \begin{center}
% %\rotatebox{-90}
% {
%  \includegraphics[width=0.5\textwidth]{<FigureName>}
% }
% \caption{\label{<Label>} <Caption>}
% \end{center}
% \end{figure}

\subsubsection{Goal Statements}

\noindent Given the \wss{inputs}, the goal statements are:


\subsection{Solution Characteristics Specification}


\subsubsection{Assumptions}


\subsubsection{Theoretical Models}\label{sec_theoretical}



\subsubsection{General Definitions}\label{sec_gendef}


\subsubsection*{Detailed derivation of simplified rate of change of temperature}

\wss{This may be necessary when the necessary information does not fit in the
  description field.} 

\subsubsection{Data Definitions}\label{sec_datadef}


\subsubsection{Instance Models} \label{sec_instance}    

\subsubsection*{Derivation of ...}

\wss{May be necessary to include this subsection in some cases.}

\subsubsection{Data Constraints} \label{sec_DataConstraints}    

\subsubsection{Properties of a Correct Solution} \label{sec_CorrectSolution}

\noindent
A correct solution must exhibit \wss{fill in the details}

\section{Requirements}

This section provides the functional requirements, the business tasks that the
software is expected to complete, and the nonfunctional requirements, the
qualities that the software is expected to exhibit.

\subsection{Functional Requirements}


\subsection{Nonfunctional Requirements}

\wss{List your nonfunctional requirements.  You may consider using a fit
  criterion to make them verifiable.}

\section{Likely Changes}    


\section{Traceability Matrices and Graphs}


\wss{You will have to modify these tables for your problem.}

\newpage

\bibliographystyle {ieeetr}
%\bibliography {../../ReferenceMaterial/References}

\newpage

\section{Appendix}

\wss{Your report may require an appendix.  For instance, this is a good point to
show the values of the symbolic parameters introduced in the report.}

\subsection{Symbolic Parameters}


\end{document}